Ce projet d'application s'inscrivant dans un travail de groupe, nous avons, dès la conception de l'application, puis lors de son développement, attaché une grande importance à l'organisation et à la répartition des tâches.

\section{Gestionnaire de version Git}
D'un point de vue technique, cette préoccupation d'organisation s'est traduite par l'utilisation du gestionnaire de version Git afin de pourvoir non seulement avoir des copies de sauvegarde de notre code sources mais également pour pouvoir travailler en parallèle les uns des autres grâce au système de branche.

L'utilisation de cet outils nous grandement facilité la tâche lors de la répartition des tâches en nous permettant de gérer indépendamment les différentes partie de l'application.

\section{Organisation modulaire}
Toujours dans le but de pourvoir nous répartir les tâches le plus facilement possible, nous avons essayé de rendre le développement de l'application le plus modulaire possible. Ainsi, nous avons séparé au maximum les modules liés à la gestion du rendu 3D des modules liés au calcul des solutions.

Enfin, dans le but d'unifier notre code sources, nous avons établies des conventions communes quant à la nomenclature des variables et des fonctions.