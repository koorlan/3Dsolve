Véritables défis pour leurs utilisateurs, les casse-tête sont des objets de divertissement axés sur la réflexion, la logique ou les mathématiques. Ces jeux individuels se déclinent en plusieurs catégories, observant divers degrés de complexité et étant plus ou moins ludiques. Parmi ces grandes familles, on trouve celle des casse-tête mécaniques dont les origines remontent jusqu’à l’un des plus grands mathématiciens de l’antiquité, Archimède. Fort de son goût pour les énigmes, il s’est employé à transposer les pratiques mathématiques à des problèmes récréatifs, donnant entre-autre naissance au Syntémachion, le premier des casse-tête géométrico-mécanique. Puis l’histoire de ces casse-tête a jalonné les siècles et le monde, pour enfin connaître un développement accru à la fin du XVIIIe siècle. En Occident, ce développement a notamment été encouragé par la publication de Puzzles old and new par le Prof. Hoffmann et, encore une fois, pour sa résonance particulière dans le milieu des mathématiques. Les XIXe et XXe siècles seront marqués respectivement par l’invention des désormais célèbres Tangram (Chine) et Rubik’s Cube (Hongrie). \newline

À la lumière des algorithmes et des programmes informatiques, ces casse-tête prennent une autre dimension. Ils ouvrent la porte à de nouveaux enjeux, de nouveaux défis. C’est ainsi que l’on dénombra 536 solutions au casse-tête inventé par Archimède et que l’on explora les solutions émanant des 4.3 x 1019  positions de départ du Rubik’s Cube. De la fasciation suscitée par ces casse-tête est née une multitude d’algorithmes puis d’applications capables de les résoudre et d’en calculer toutes les solutions. Ces travaux ont été rendus possibles par l’avancée des technologies informatiques, le déploiement des architectures multi-cœurs et du calcul partagé.  \newline

C’est dans cette optique que nous nous sommes attelés à développer une application autour d’un casse-tête mécanique plutôt méconnu, le Snake Cube. Cette méconnaissance a sans doute favorisé une vision authentique du problème, nous avons apprivoisé le Snake Cube et nous en avons compris les rouages au fur et à mesure que l’analyse préliminaire s’étayait sur le papier. Grâce à la combinaison de préceptes algorithmiques et mathématiques, nous avons déterminé le schéma de résolution du casse-tête. Ensuite l’application s’est prolongée avec le développement d’une interface homme-machine fluide et l’implémentation du jeu de l’utilisateur, lui permettant de manier lui-même un Snake Cube virtuel. À la fin de ce projet, il était nécessaire de fournir des résultats fidèles et complets quant aux solutions proposées et ce dans des temps de calculs acceptables. Mais outre ces impératifs de résultats, nous avons gardé à l’esprit que cette application devait être jouable et réellement utilisable. Il nous paraissait important qu’une fois fini, ce projet pouvait et devait en intéresser d‘autres que nous. Puisque finalement, qui n’a jamais capitulé devant un casse-tête ou une énigme difficile à résoudre ? À ceux-là et aux autres, férus de casse-tête mécaniques ou d’algorithmes, nous leur proposons une solution au Snake Cube, et bien plus encore. \newline       
	
