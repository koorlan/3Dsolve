\section{Résoudre par le calcul un Snake Cube}
Dans un premier temps, notre objectif consiste en l’implémentation d’une application capable de résoudre des Snake Cubes. L’utilisateur fournit au résolveur deux paramètres essentiels : la définition d’un serpent initial ainsi qu’un volume final. Ensuite l’application automatise la tâche de recherche pour présenter à l’utilisateur la liste de solutions correspondant au problème. Les solutions devront être présentées de manières claires avec, pour chacune d’entre elles, la possibilité pour l’utilisateur de la visionner pas-à-pas ou bien encore de revenir en arrière. Une première contrainte évidente liée à cet objectif concerne le temps nécessaire au calcul. Ainsi, il va de soi que le résolveur devra à la fois s’employer à utiliser des méthodes de calcul pertinentes et être capable de ne pas rechercher des solutions redondantes. Cette redondance traduit en fait le caractère symétrique de plusieurs solutions. 

\section{Proposer à l'utilisateur de résoudre lui-même le Snake Cube}
Dans un second temps, nous proposons à l’utilisateur un mode interactif où celui-ci pourra tenter de résoudre virtuellement le casse-tête. En termes de contraintes, ce versant de l’application nécessite très peu de calcul. Néanmoins, il exige une réflexion accrue concernant la représentation graphique. En effet, il faut ici proposer au joueur un environnement en 3 dimensions à la fois clair et maniable ainsi qu’un panel de fonctionnalités rendant l’application intuitive.