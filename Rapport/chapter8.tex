Toujours dans l’objectif de fournir une application la plus simple a utiliser nous avons décider de dédier une partie du développement au support des différentes plateformes. Bien que compliquée, notre application peut être compilé et exécutée sur les 3 principaux systèmes d’exploitation du marché à savoir :

\begin{itemize}
 \item GNU Linux
 \item Microsoft Windows
 \item OSX Apple
\end{itemize}

Pour la compilation, diverses bibliothèques nécessitent d’être installées (voir le README fournis avec les sources). Le Makefile détecte l’OS utilisé et adapte la chaine de compilation pour la plateforme adéquate. Nous avons également utilisé les commandes préprocesseur de détection de système d’exploitation afin d’avoir un fichier source commun aux différentes plateformes de compilation.

Par ailleurs l’application peut être lancée sur des machines n’ayant pas ces bibliothèques installées grâce à la compilation statique. Les résultats sont encourageants, sur une vingtaine de machines testées seules quelques une ont échouées au lancement du programme. L’application est donc prête a un déploiement et une distribution multi-plateforme.

La compatibilité graphique a été un vrai défi puisque nous avons travaillé avec certains types de données d’OpenGL 3 au début du projet, il s’est avéré malheureusement que certaines de nos machines nécessaires aux tests de portabilités n’acceptaient que des fonctions jusqu’à OpenGL 2.1 et certaines extensions. 

Le développement et l’adaptation des techniques de rendu a donc été adapté pour répondre a ces contraintes de versions.