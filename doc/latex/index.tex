Ce programme résoud, de manière calculatoire, des \char`\"{}\-Snake Cube\char`\"{}. La méthode de recherche de solution est itérative et utilise un arbre n-\/aire.

Le principe est le suivant \-:
\begin{DoxyEnumerate}
\item Un algorithme de recherche de symétrie élime les points de départ symétiques afin de réduire au maximum le temps de calcul.
\item Un autre algorithme essaie tous les chemins (combinaison de placement des éléments du snake) qu'il est possible de générer depuis les points trouvés à l'étape 1.
\item Les chemins qui aboutissent (qui permettent de placer tous les éléments du snake de manière valide) sont des solutions.
\item Les solutions sont ensuite présentées à l'utilisateur dans une scène de rendu en 3 dimentions. L'utilisateur peut regarder pas-\/à-\/pas la résolution du casse-\/tête.
\end{DoxyEnumerate}

\begin{DoxyAuthor}{Auteurs}
Lisa Aubry \href{mailto:lisa.aubry@insa-cvl.fr}{\tt lisa.\-aubry@insa-\/cvl.\-fr}, Alban Chazot \href{mailto:alban.chazot@insa-cvl.fr}{\tt alban.\-chazot@insa-\/cvl.\-fr}, Korlan Colas \href{mailto:korlan.colas@insa-cvl.fr}{\tt korlan.\-colas@insa-\/cvl.\-fr}, Anthony Gourd \href{mailto:anthony.gourd@insa-cvl.fr}{\tt anthony.\-gourd@insa-\/cvl.\-fr} 
\end{DoxyAuthor}
\begin{DoxyDate}{Date}
Juin 2015
\end{DoxyDate}
Projet tutoré par Patrice Clemente \href{mailto:patrice.clemente@insa-cvl.fr}{\tt patrice.\-clemente@insa-\/cvl.\-fr}

I\-N\-S\-A Centre Val de Loire \-: Année 2014-\/2015

Promotion 2017 